\begin{frame}
    \frametitle{Methodology}
    \framesubtitle{Main Model}
    \begin{itemize}
        \item \textbf{Step I:} By CZ c and year y: Construct native average wages, native unemployment and labor force participation rates
    \end{itemize}
    \begin{itemize}
        \item \textbf{Step II:} Construct immigrant inflow: 
        \item \[x_c = \frac{1}{N_{c,1980}} \left(I_{c,2007} - I_{c,1980}\right)\]
    \end{itemize}
    \begin{itemize}
        \item $N_{c,1980} = $ total population of $c$ in year 1980
        \item $I_{c,y} = $ population of immigrants in $c$ in year $y$
    \end{itemize}
\end{frame}
%% ------------------------------------------------------------------------
\begin{frame}
    \frametitle{Methodology}
    \framesubtitle{Main Model} continued
    \begin{itemize}
        \item \textbf{Step III:} Construct instrument I:
        \[z_c = \frac{1}{N_{c,1980}} \sum_s f^s_{c,1980} \left(I^s_{2007} - I^s_{1980}\right)\]
    \end{itemize}
    \begin{itemize}
        \item $I^s_y = $ number of immigrants from source region $s$ in the US in year $y$
        \item $f^s_{c,1980} = \frac{I^s_{c,1980}}{I^s_{1980}}$ share of immigrants from $s$ who are in $c$ in year 1980
    \end{itemize}
    \begin{itemize}
        \item \textbf{Step IIIa:} Construct instrument II:
        \[z_c^{\text{alt}} = \frac{1}{I_{c,1980}} \sum_s f^s_{c,1980} \left(I^s_{2007} - I^s_{1980}\right)\]
    \end{itemize}
    \begin{itemize}
        \item \textbf{Step IV:} 
        \item Using 2SLS, project changes in CZ outcomes (percentage point for unemployment and LFP; percent or log for wage) on immigrant inflow
        \item instrument for, xc, with either, zc, or its alternative, zalt, depending on which has a
        stronger first stage when you include the controls
        \item include controls (like in Autor, Dorn, and Hanson) measured in 1980 data: key control is the share of the population that is immigrant in 1980
    \end{itemize}
\end{frame}

%% ------------------------------------------------------------------------

\begin{frame}
    \frametitle{Methodology}
    \framesubtitle{Main Model}

    We define the random variable $Y$ as the change in a specific outcome for U.S. natives, $X$ as the Immigrant Inflow, and $Z$ as the instrument for $X$, all variables in regional level. The sample $\left\{Y_c, X_c, Z_c\right\}_{c = 1}^{722}$ consists of 722 CZs across the United States.
    
    Due to the potential endogeneity of $X$, the structural model is proposed as follows:

    \begin{align}
        Y_c = \alpha +  \beta X_c + \Vec{W_c}'\Vec{\gamma} + u_c \\
        X_c = \phi + \xi Z_c + \Vec{W_c}'\Vec{\theta} + \nu_c\\
    \end{align}
    \begin{align}
        \E{u_c | X_c} &\neq 0 \\
        \cov{X_c, Z_c} &\neq 0 \\
        \E{u_c | Z_c} &= \E{\nu_c | Z_c} = 0
    \end{align}

    Where $W$ is a vector of controls.
    
    This model is estimated using 2SLS, correcting inference for heteroskedasticity and autocorrelation with clustered robust standard errors, grouped by state.
\end{frame}

%% ------------------------------------------------------------------------

\begin{frame}
    \frametitle{Methodology}
    \framesubtitle{Instrumental Relevance}
    We are interested in evaluating the relationship between the instrument $Z$ and the endogenous variable $X$, specifically $\cov{X_c, Z_c} \neq 0$, given the control variables $W$. To do so, we use the auxiliary regression:

    \begin{align}
        r_{X,c} = \psi r_{Z,c} + \omega_{c}
        \label{eq:auxiliary}
    \end{align}
    
    where $r_{X,c}$ and $r_{Z,c}$ are the orthogonal components of $X$ and $Z$, respectively, defined as:
    
    \begin{align}
        X_c = a_0 + \Vec{W_c}'\Vec{a_1} + r_{X,c} \\
        Z_c = b_0 + \Vec{W_c}'\Vec{b_1} + r_{Z,c}
    \end{align}
    
\end{frame}

%% ------------------------------------------------------------------------

\begin{frame}
    \frametitle{Methodology}
    \framesubtitle{Instrumental Relevance}
    
    The null hypothesis that the instrument is irrelevant ($\psi = 0$) is rejected if the $F_{partial}$ statistic exceeds 10\footnote{Staiger \& Stock (1997)}. Alternatively, this can be tested using a $\chi^2$ distribution with one degree of freedom\footnote{Montiel Olea \& Pflueger (2013)}, as we have a single endogenous variable and a single instrument.
    
    The $F_{partial}$ statistic is defined as:
    
    \begin{align}
        F_{partial} = \frac{R^2}{\frac{1-R^2}{n - 1}}
    \end{align}
    
    where $R^2$ is the coefficient of determination from the auxiliary regression \ref{eq:auxiliary}, and $n$ is the number of observations, which in this case is 722.

    \emph{Based on this criterion, the instrument that is relevant for the endogenous variable is selected.}

\end{frame}
