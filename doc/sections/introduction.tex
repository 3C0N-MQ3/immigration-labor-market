\begin{frame}

    \begin{block}{Research Question}
    What is the impact of large inflows of immigrants on inequality across regions in the United States?
    \end{block}
    
    \begin{block}{Methodological Approach} 
    To be able to address the Research Question, the total U.S. population in 1980 —with its respective immigrant rate from source— was used as a reference and compared to the estimated immigrant inflow for 2007 across various Commuting Zones (CZ).
    The application of Two-Step Least Squares (2SLS), projected changes in CZ outcomes and the use of the Standard Card Instrument (SCI) and Autor, Dorn, and Hanson (ADH) controls allowed for determining the effects on the variables Native Wages, Native Unemployment, and Labor Force Participation (LFP).
    Based on this, the three elements to study inequality were defined: 1) population, 2) variable, and 3) measure, in alignment with the Income Inequality Course.
    \end{block}
    
    \begin{block}{Key Findings} 
    It was found that large immigrant inflows do not have a statistically significant impact on the evaluated variables. Based on this, it cannot be concluded that large immigrant inflows increase the inequality gap in the United States during the study period.
    \end{block}
    
\end{frame}
