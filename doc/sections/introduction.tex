\begin{frame}

    \begin{block}{Research Question}
    How do large immigrant inflows affect regional inequality in the United States?
    \end{block}
    
    \begin{block}{Methodological Approach} 
    This study uses U.S. census data from 1980 and 2007 to evaluate the impact of changes in immigrant inflows on native wages and employment across Commuting Zones (CZs). \\

    To address potential endogeneity in immigrant inflows, a Two-Step Least Squares (2SLS) approach was implemented, with the instrumental variable chosen between the Standard Card Instrument (SCI) and the Predicted Immigration Growth Rate.
    \end{block}
    
    \begin{block}{Key Findings} 
        The results indicate that changes in immigrant inflows do not significantly affect wages but have a notable negative impact on employment and labor force participation.
    \end{block}
    
    
\end{frame}

